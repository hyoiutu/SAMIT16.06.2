%\documentclass[a4j,twocolumn, dvipdfmx]{jsarticle} % 二段組の構成にする
%\documentclass[a4j,notitlepage, dvipdfmx]{jsarticle} % タイトルだけのページを作らない
\documentclass[a4j,titlepage,dvipdfmx]{jsarticle}   % タイトルだけのページを作る
\usepackage{amsmath} % align環境を使う為のパッケージ
\usepackage{graphicx} % includegraphicsコマンドwお使うためのパッケージ

\title{レポートのタイトル}
\author{レポートの作者 \and
        レポートの作者その2 \and
        レポートの作者その3}
\date{XXXX年XX月XX日}
%\date{\today} 今日の日付が入る

\begin{document}
  \maketitle % タイトル,著者,日付が出力される
  \tableofcontents % 目次が出力される
  \clearpage % 目次を独立したページに出力する

  \section{大節}
  \subsection{小節}
  \subsubsection{小々節}
  このように節は大きいものから小さいものまである.
  \section*{大節(節番号なし)}
  \subsection*{小節(節番号なし)}
  \subsubsection*{小々節(節番号なし)}
  コマンドの後に*を付けると節番号が出力されない.
  \section{箇条書き}
  \subsection{itemize}
  箇条書きの代表的な環境はitemize環境である.
  \begin{itemize}
    \item hoge
    \item fuga
    \item foo
    \begin{itemize}
      \item 入れ子
      \item 構造も
      \item この通り
    \end{itemize}
  \end{itemize}
  \subsection{enumerate}
  番号をつけるにはenumerate環境を使う.
  \begin{enumerate}
      \item 番号が
      \item 付いている
      \item はず
  \end{enumerate}
  \subsection{description}
  自由に項目を付けるにはdescription環境を使う.
  \begin{description}
      \item[□] 様々な
      \item[△] 項目に
      \item[◇] なっているはず
  \end{description}
  \section{数式を使う}
  \subsection{インライン数式モード}
  文章中に数式を埋め込む場合はインライン数式モードを使う.この式$f(t) = \sum_{n=-\infty}^{\infty} F_n e^{int}$は
  フーリエ逆変換の離散スペクトルの場合であり,
  次の式$f(t) = \frac{1}{2\pi}\int_{-\infty}^{\infty}F(\omega)e^{i\omega t} d\omega$は
  連続スペクトルの場合である.
  \subsection{ディスプレイ数式モード}
  数式だけの行を作りたいのであればディスプレイ数式モードを使う.
  \subsubsection{数式が1行の場合}
  \begin{equation}
    f(t) = \sum_{n=-\infty}^{\infty} F_n e^{int}
  \end{equation}
  \subsubsection{数式が複数行の場合}
  \begin{align*} % *をなくすと数式番号が末尾につく
    \intertext{離散スペクトルの場合は}
    f(t) &= \sum_{n=-\infty}^{\infty} F_n e^{int} \\
    \intertext{と表され,連続スペクトルの場合は}
    f(t) &= \frac{1}{2\pi}\int_{-\infty}^{\infty}F(\omega)e^{i\omega t} d\omega \\
    \intertext{のように表される.}
  \end{align*}
  \section{図表の挿入}
  \subsection{図}
  図はjpg,png,bmp,gif,eps,pdfなどが入る.レポートの図であればpdfをおすすめする.
  \includegraphics[width=5cm,height=5cm]{test.pdf}
\end{document}
