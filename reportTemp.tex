%\documentclass[a4j,twocolumn, dvipdfmx]{jsarticle} % 二段組の構成にする
%\documentclass[a4j,notitlepage, dvipdfmx]{jsarticle} % タイトルだけのページを作らない
\documentclass[a4j,titlepage,dvipdfmx]{jsarticle}   % タイトルだけのページを作る
\usepackage{amsmath} % align環境を使う為のパッケージ
\usepackage{graphicx} % includegraphicsコマンドwお使うためのパッケージ
\usepackage{here} % Hオプションを使うため
\usepackage[cache=false]{minted} % mintedを使うため
\usepackage{listings} % listingを使うため
\usepackage{jlisting} % listingを日本語に対応させるため
\usepackage{color} % listingでシンタックスハイライトするため
\usepackage{itembkbx} % breakitemboxを使うため
\usepackage{verbatim} % verbatiminputを使うため


\newminted[myMinted]{c}{%
  linenos,
  mathescape,
  numbersep=5pt,
  frame=lines,
  framesep=2mm
} % あらかじめmintedの設定を記述しておく
\newmintedfile[myMintedfile]{c}{
    linenos,
    mathescape,
    numbersep=5pt,
    frame=lines,
    framesep=2mm
}

\lstset{%
  language={C},
  basicstyle={\small},%
  identifierstyle={\small},%
  commentstyle={\small\itshape\color[rgb]{0,0.5,0}},%
  keywordstyle={\small\bfseries\color[rgb]{0,0,1}},%
  ndkeywordstyle={\small},%
  stringstyle={\small\ttfamily\color[rgb]{1,0,1}},
  frame={tb},
  breaklines=true,
  columns=[l]{fullflexible},%
  numbers=left,%
  xrightmargin=0zw,%
  xleftmargin=3zw,%
  numberstyle={\scriptsize},%
  stepnumber=1,
  numbersep=1zw,%
  lineskip=-0.5ex%
} % あらかじめlistingの設定を書いておく

\title{レポートのタイトル}
\author{レポートの作者 \and
        レポートの作者その2 \and
        レポートの作者その3}
\date{XXXX年XX月XX日}
%\date{\today} 今日の日付が入る

\begin{document}

  \maketitle % タイトル,著者,日付が出力される
  \tableofcontents % 目次が出力される
  \clearpage % 目次を独立したページに出力する

  \section{大節}
  \subsection{小節}
  \subsubsection{小々節}
  このように節は大きいものから小さいものまである.

  \section*{大節(節番号なし)}
  \subsection*{小節(節番号なし)}
  \subsubsection*{小々節(節番号なし)}
  コマンドの後に*を付けると節番号が出力されない.

  \section{箇条書き}
  \subsection{itemize}
  箇条書きの代表的な環境はitemize環境である.
  \begin{itemize}
    \item hoge
    \item fuga
    \item foo
    \begin{itemize}
      \item 入れ子
      \item 構造も
      \item この通り
    \end{itemize}
  \end{itemize}

  \subsection{enumerate}
  番号をつけるにはenumerate環境を使う.
  \begin{enumerate}
      \item 番号が
      \item 付いている
      \item はず
  \end{enumerate}

  \subsection{description}
  自由に項目を付けるにはdescription環境を使う.
  \begin{description}
      \item[□] 様々な
      \item[△] 項目に
      \item[◇] なっているはず
  \end{description}

  \section{数式を使う}
  \subsection{インライン数式モード}
  文章中に数式を埋め込む場合はインライン数式モードを使う.この式$f(t) = \sum_{n=-\infty}^{\infty} F_n e^{int}$は
  フーリエ逆変換の離散スペクトルの場合であり,
  次の式$f(t) = \frac{1}{2\pi}\int_{-\infty}^{\infty}F(\omega)e^{i\omega t} d\omega$は
  連続スペクトルの場合である.

  \subsection{ディスプレイ数式モード}
  数式だけの行を作りたいのであればディスプレイ数式モードを使う.

  \subsubsection{数式が1行の場合}
  \begin{equation}
    f(t) = \sum_{n=-\infty}^{\infty} F_n e^{int}
  \end{equation}

  \subsubsection{数式が複数行の場合}
  \begin{align*} % *をなくすと数式番号が末尾につく
    \intertext{離散スペクトルの場合は}
    f(t) &= \sum_{n=-\infty}^{\infty} F_n e^{int} \\
    \intertext{と表され,連続スペクトルの場合は}
    f(t) &= \frac{1}{2\pi}\int_{-\infty}^{\infty}F(\omega)e^{i\omega t} d\omega \\
    \intertext{のように表される.}
  \end{align*}

  \section{図表の挿入}
  \subsection{図}
  図はjpg,png,bmp,gif,eps,pdfなどが入る.レポートの図であればpdfをおすすめする.
  \begin{figure}[H]
    \centering
    \includegraphics[width=\textwidth]{test.pdf}
    \caption{$\frac{\sin{x}}{\frac{\cos{x}}{\tan{x}}}$}
    \label{fig:tri}
  \end{figure}
  図\ref{fig:tri}は参照されている図である.

  \subsection{表}
  \begin{table}[H]
    \centering
    \caption{謎の表}
    \label{tab:testTab}
    \begin{tabular}{|l|l|r|}
      \hline
      Animal      & Description  & Price (\$) \\ \hline
      Gnat        & per gram     & 13.65      \\ \hline
                  & each         & 0.01       \\ \hline
      Gnu         & stuffed      & 92.50      \\ \hline
      Emu         & stuffed      & 33.33      \\ \hline
      Armadillo   & frozen       & 8.99       \\ \hline
    \end{tabular}
  \end{table}
  表\ref{tab:testTab}である.このように参照できる.

  \subsection{複数の表や図を並べて表示する}
  \begin{figure}[H]
    \begin{tabular}{cc}
      \begin{minipage}{0.5\hsize}
        \centering
        \includegraphics[width=\textwidth]{sinFig.pdf}
        \caption{$\sin{x}$}
        \label{fig:sin}
      \end{minipage}
      \begin{minipage}{0.5\hsize}
        \centering
        \includegraphics[width=\textwidth]{cosFig.pdf}
        \caption{$\cos{x}$}
        \label{fig:cos}
      \end{minipage}
    \end{tabular}
  \end{figure}

  \section{ソースコードを載せる}
  \subsection{minted}
  \begin{myMinted}
    #include<stdio.h>

    int main(void){
      printf("Hello World!!\");
      return 0;
    }
  \end{myMinted}

  \subsection{listing}
  \begin{lstlisting}[caption=listingのテスト,label=listTest]
    #include<stdio.h>

    int main(void){
      printf("Hello World!!\");
      return 0;
    }
  \end{lstlisting}

  \subsection{breakitembox+verbatim}
  \begin{breakitembox}[l]{見出し}
    \begin{verbatim}
        #include<stdio.h>
        int main(void){
            printf(Hello World!\n);
            return 0;
        }
    \end{verbatim}
\end{breakitembox}

\section{ソースコードをファイルから載せる}
\subsection{minted}
\myMintedfile{test.c}

\subsection{listing}
\lstinputlisting[caption=aaa,label=aaa]{test.c}

\subsection{breakitembox+verbatiminput}
\begin{breakitembox}[l]{見出し}
   \verbatiminput{test.c}
\end{breakitembox}

\end{document}
