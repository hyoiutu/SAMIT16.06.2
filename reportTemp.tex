%\documentclass[a4j,twocolumn, dvipdfmx]{jsarticle} % 二段組の構成にする
%\documentclass[a4j,notitlepage, dvipdfmx]{jsarticle} % タイトルだけのページを作らない
\documentclass[a4j,titlepage,dvipdfmx]{jsarticle}   % タイトルだけのページを作る
\usepackage{amsmath} % align環境を使う為のパッケージ
\usepackage{graphicx} % includegraphicsコマンドwお使うためのパッケージ
\usepackage{here} % Hオプションを使うため
\usepackage[cache=false]{minted} % mintedを使うため
\usepackage{listings} % listingを使うため
\usepackage{jlisting} % listingを日本語に対応させるため
\usepackage{color} % listingでシンタックスハイライトするため
\usepackage{itembkbx} % breakitemboxを使うため
\usepackage{verbatim} % verbatiminputを使うため
\usepackage{tikz} % texに直接図を埋め込むため
\usepackage{gnuplot-lua-tikz} % texに直接図を埋め込むため

\newminted[myMinted]{c}{%
  linenos,
  mathescape,
  numbersep=5pt,
  frame=lines,
  framesep=2mm
} % あらかじめmintedの設定を記述しておく
\newmintedfile[myMintedfile]{c}{
    linenos,
    mathescape,
    numbersep=5pt,
    frame=lines,
    framesep=2mm
} % あらかじめmintedの設定を記述しておく
\newminted{shell}{%
  linenos,
  mathescape,
  numbersep=5pt,
  frame=lines,
  framesep=2mm
} % あらかじめmintedの設定を記述しておく

\lstset{%
  language={C},
  basicstyle={\small},%
  identifierstyle={\small},%
  commentstyle={\small\itshape\color[rgb]{0,0.5,0}},%
  keywordstyle={\small\bfseries\color[rgb]{0,0,1}},%
  ndkeywordstyle={\small},%
  stringstyle={\small\ttfamily\color[rgb]{1,0,1}},
  frame={tb},
  breaklines=true,
  columns=[l]{fullflexible},%
  numbers=left,%
  xrightmargin=0zw,%
  xleftmargin=3zw,%
  numberstyle={\scriptsize},%
  stepnumber=1,
  numbersep=1zw,%
  lineskip=-0.5ex%
} % あらかじめlistingの設定を書いておく

\title{レポートのタイトル}
\author{レポートの作者 \and
        レポートの作者その2 \and
        レポートの作者その3}
\date{XXXX年XX月XX日}
%\date{\today} 今日の日付が入る

\begin{document}

  \maketitle % タイトル,著者,日付が出力される
  \tableofcontents % 目次が出力される
  \clearpage % 目次を独立したページに出力する

  \section{大節}
  \subsection{小節}
  \subsubsection{小々節}
  このように節は大きいものから小さいものまである.

  \section*{大節(節番号なし)}
  \subsection*{小節(節番号なし)}
  \subsubsection*{小々節(節番号なし)}
  コマンドの後に*を付けると節番号が出力されない.

  \section{箇条書き}
  \subsection{itemize}
  箇条書きの代表的な環境はitemize環境である.
  \begin{itemize}
    \item hoge
    \item fuga
    \item foo
    \begin{itemize}
      \item 入れ子
      \item 構造も
      \item この通り
    \end{itemize}
  \end{itemize}

  \subsection{enumerate}
  番号をつけるにはenumerate環境を使う.
  \begin{enumerate}
      \item 番号が
      \item 付いている
      \item はず
  \end{enumerate}

  \subsection{description}
  自由に項目を付けるにはdescription環境を使う.
  \begin{description}
      \item[□] 様々な
      \item[△] 項目に
      \item[◇] なっているはず
  \end{description}

  \section{数式を使う}
  \subsection{インライン数式モード}
  文章中に数式を埋め込む場合はインライン数式モードを使う.この式$f(t) = \sum_{n=-\infty}^{\infty} F_n e^{int}$は
  フーリエ逆変換の離散スペクトルの場合であり,
  次の式$f(t) = \frac{1}{2\pi}\int_{-\infty}^{\infty}F(\omega)e^{i\omega t} d\omega$は
  連続スペクトルの場合である.

  \subsection{ディスプレイ数式モード}
  数式だけの行を作りたいのであればディスプレイ数式モードを使う.

  \subsubsection{数式が1行の場合}
  \begin{equation}
    f(t) = \sum_{n=-\infty}^{\infty} F_n e^{int}
  \end{equation}

  \subsubsection{数式が複数行の場合}
  \begin{align*} % *をなくすと数式番号が末尾につく
    \intertext{離散スペクトルの場合は}
    f(t) &= \sum_{n=-\infty}^{\infty} F_n e^{int} \\
    \intertext{と表され,連続スペクトルの場合は}
    f(t) &= \frac{1}{2\pi}\int_{-\infty}^{\infty}F(\omega)e^{i\omega t} d\omega \\
    \intertext{のように表される.}
  \end{align*}

  \section{図表の挿入}
  \subsection{図}
  図はjpg,png,bmp,gif,eps,pdfなどが入る.レポートの図であればベクタ画像%
  \footnote{いくら拡大しても荒くならない.色のドット情報で表されるのがラスタ画像(jpg,png,bmp,gifなど)で,
  線の情報で表されるのがベクタ画像(eps,pdf,svgなど)}でありながら軽いpdfをおすすめする.
  \begin{figure}[H]
    \centering
    \includegraphics[width=\columnwidth]{test.pdf}
    \caption{$\frac{\sin{x}}{\frac{\cos{x}}{\tan{x}}}$}
    \label{fig:tri}
  \end{figure}
  図\ref{fig:tri}は参照されている図である.

  \subsection{表}
  \begin{table}[H]
    \centering
    \caption{謎の表}
    \label{tab:testTab}
    \begin{tabular}{|l|l|r|}
      \hline
      Animal      & Description  & Price (\$) \\ \hline
      Gnat        & per gram     & 13.65      \\ \hline
                  & each         & 0.01       \\ \hline
      Gnu         & stuffed      & 92.50      \\ \hline
      Emu         & stuffed      & 33.33      \\ \hline
      Armadillo   & frozen       & 8.99       \\ \hline
    \end{tabular}
  \end{table}
  表\ref{tab:testTab}である.このように参照できる.

  \subsection{複数の表や図を並べて表示する}
  \begin{figure}[H]
    \begin{tabular}{cc}
      \begin{minipage}{0.5\hsize}
        \centering
        \includegraphics[width=\columnwidth]{sinFig.pdf}
        \caption{$\sin{x}$}
        \label{fig:sin}
      \end{minipage}
      \begin{minipage}{0.5\hsize}
        \centering
        \includegraphics[width=\columnwidth]{cosFig.pdf}
        \caption{$\cos{x}$}
        \label{fig:cos}
      \end{minipage}
    \end{tabular}
  \end{figure}

  \section{ソースコードを載せる}
  \begin{listing}[htbp]
    \caption{初回に必要なコマンド}
    \begin{shellcode}
      sudo apt install python-pygments
    \end{shellcode}
  \end{listing}
  \subsection{minted}
  \begin{myMinted}
    #include<stdio.h>

    int main(void){
      printf("Hello World!!\");
      return 0;
    }
  \end{myMinted}

  \subsection{listing}
  \begin{lstlisting}[caption=listingのテスト,label=listTest]
    #include<stdio.h>

    int main(void){
      printf("Hello World!!\");
      return 0;
    }
  \end{lstlisting}

  \subsection{breakitembox+verbatim}
  \begin{breakitembox}[l]{見出し}
    \begin{verbatim}
        #include<stdio.h>
        int main(void){
            printf(Hello World!\n);
            return 0;
        }
    \end{verbatim}
\end{breakitembox}

\section{ソースコードをファイルから載せる}
\subsection{minted}
\myMintedfile{test.c}

\subsection{listing}
\lstinputlisting[caption=aaa,label=aaa]{test.c}

\subsection{breakitembox+verbatiminput}
\begin{breakitembox}[l]{見出し}
   \verbatiminput{test.c}
\end{breakitembox}

\section{texにgnuplotの図を直接記述する}
\begin{shellcode}
  set title 'test graph'
  set xlabel 'x'
  set ylabel 'f(x)'
  set xrange [-5:5]
  set yrange [-0.1:1.1]
  plot 1/(1+exp(-x))
  set terminal tikz
  set output 'tikztest.tex'
  replot
  quit
\end{shellcode}
\centering
\resizebox{\columnwidth}{!}{\begin{tikzpicture}[gnuplot]
%% generated with GNUPLOT 4.6p6 (Lua 5.1; terminal rev. 99, script rev. 100)
%% 2016年07月11日 13時49分37秒
\path (0.000,0.000) rectangle (12.500,8.750);
\gpcolor{color=gp lt color border}
\gpsetlinetype{gp lt border}
\gpsetlinewidth{1.00}
\draw[gp path] (1.504,1.555)--(1.684,1.555);
\draw[gp path] (11.947,1.555)--(11.767,1.555);
\node[gp node right] at (1.320,1.555) { 0};
\draw[gp path] (1.504,2.695)--(1.684,2.695);
\draw[gp path] (11.947,2.695)--(11.767,2.695);
\node[gp node right] at (1.320,2.695) { 0.2};
\draw[gp path] (1.504,3.835)--(1.684,3.835);
\draw[gp path] (11.947,3.835)--(11.767,3.835);
\node[gp node right] at (1.320,3.835) { 0.4};
\draw[gp path] (1.504,4.975)--(1.684,4.975);
\draw[gp path] (11.947,4.975)--(11.767,4.975);
\node[gp node right] at (1.320,4.975) { 0.6};
\draw[gp path] (1.504,6.115)--(1.684,6.115);
\draw[gp path] (11.947,6.115)--(11.767,6.115);
\node[gp node right] at (1.320,6.115) { 0.8};
\draw[gp path] (1.504,7.255)--(1.684,7.255);
\draw[gp path] (11.947,7.255)--(11.767,7.255);
\node[gp node right] at (1.320,7.255) { 1};
\draw[gp path] (2.548,0.985)--(2.548,1.165);
\draw[gp path] (2.548,7.825)--(2.548,7.645);
\node[gp node center] at (2.548,0.677) {-4};
\draw[gp path] (4.637,0.985)--(4.637,1.165);
\draw[gp path] (4.637,7.825)--(4.637,7.645);
\node[gp node center] at (4.637,0.677) {-2};
\draw[gp path] (6.726,0.985)--(6.726,1.165);
\draw[gp path] (6.726,7.825)--(6.726,7.645);
\node[gp node center] at (6.726,0.677) { 0};
\draw[gp path] (8.814,0.985)--(8.814,1.165);
\draw[gp path] (8.814,7.825)--(8.814,7.645);
\node[gp node center] at (8.814,0.677) { 2};
\draw[gp path] (10.903,0.985)--(10.903,1.165);
\draw[gp path] (10.903,7.825)--(10.903,7.645);
\node[gp node center] at (10.903,0.677) { 4};
\draw[gp path] (1.504,7.825)--(1.504,0.985)--(11.947,0.985)--(11.947,7.825)--cycle;
\node[gp node center,rotate=-270] at (0.246,4.405) {$f(x)$};
\node[gp node center] at (6.725,0.215) {$x$};
\node[gp node center] at (6.725,8.287) {シグモイド関数のグラフ};
\node[gp node right] at (10.479,7.491) {$\frac{1}{1+e^{-x}}$};
\gpcolor{color=gp lt color 0}
\gpsetlinetype{gp lt plot 0}
\draw[gp path] (10.663,7.491)--(11.579,7.491);
\draw[gp path] (1.504,1.593)--(1.609,1.597)--(1.715,1.602)--(1.820,1.607)--(1.926,1.612)%
  --(2.031,1.618)--(2.137,1.625)--(2.242,1.632)--(2.348,1.640)--(2.453,1.649)--(2.559,1.659)%
  --(2.664,1.669)--(2.770,1.681)--(2.875,1.694)--(2.981,1.709)--(3.086,1.725)--(3.192,1.742)%
  --(3.297,1.761)--(3.403,1.782)--(3.508,1.805)--(3.614,1.831)--(3.719,1.858)--(3.825,1.889)%
  --(3.930,1.922)--(4.036,1.958)--(4.141,1.998)--(4.247,2.041)--(4.352,2.087)--(4.458,2.138)%
  --(4.563,2.193)--(4.669,2.253)--(4.774,2.317)--(4.880,2.386)--(4.985,2.461)--(5.090,2.540)%
  --(5.196,2.625)--(5.301,2.716)--(5.407,2.812)--(5.512,2.914)--(5.618,3.021)--(5.723,3.134)%
  --(5.829,3.252)--(5.934,3.374)--(6.040,3.502)--(6.145,3.633)--(6.251,3.768)--(6.356,3.906)%
  --(6.462,4.047)--(6.567,4.190)--(6.673,4.333)--(6.778,4.477)--(6.884,4.620)--(6.989,4.763)%
  --(7.095,4.904)--(7.200,5.042)--(7.306,5.177)--(7.411,5.308)--(7.517,5.436)--(7.622,5.558)%
  --(7.728,5.676)--(7.833,5.789)--(7.939,5.896)--(8.044,5.998)--(8.150,6.094)--(8.255,6.185)%
  --(8.361,6.270)--(8.466,6.349)--(8.571,6.424)--(8.677,6.493)--(8.782,6.557)--(8.888,6.617)%
  --(8.993,6.672)--(9.099,6.723)--(9.204,6.769)--(9.310,6.812)--(9.415,6.852)--(9.521,6.888)%
  --(9.626,6.921)--(9.732,6.952)--(9.837,6.979)--(9.943,7.005)--(10.048,7.028)--(10.154,7.049)%
  --(10.259,7.068)--(10.365,7.085)--(10.470,7.101)--(10.576,7.116)--(10.681,7.129)--(10.787,7.141)%
  --(10.892,7.151)--(10.998,7.161)--(11.103,7.170)--(11.209,7.178)--(11.314,7.185)--(11.420,7.192)%
  --(11.525,7.198)--(11.631,7.203)--(11.736,7.208)--(11.842,7.213)--(11.947,7.217);
\gpcolor{color=gp lt color border}
\gpsetlinetype{gp lt border}
\draw[gp path] (1.504,7.825)--(1.504,0.985)--(11.947,0.985)--(11.947,7.825)--cycle;
%% coordinates of the plot area
\gpdefrectangularnode{gp plot 1}{\pgfpoint{1.504cm}{0.985cm}}{\pgfpoint{11.9%% gnuplot variables
tikzpicture}
}


\end{document}
