\documentclass[a4j,uplatex,dvipdfmx]{beamer}
\usepackage{minted}
\usepackage{spverbatim}
\usetheme{SAMIT}
\title{latexmkの導入}
\author{FujiwaLaTeX}
\institute{室蘭工業大学工学研究科 情報電子工学系専攻}
\begin{document}
\maketitle
\begin{frame}[fragile]{\LaTeX 文書のタイプセット}
  \begin{minted}{shell}
    platex hogehoge.tex
    dvipdfmx hogehoge.dvi
    evince hogehoge.pdf
  \end{minted}
  \onslide<2>{\LARGE ソースの修正の度にコマンドを実行}
\end{frame}
\begin{frame}{latexmkとは}
  \LaTeX のソースの変更を検知し,\\
  自動的にタイプセットを行うスクリプト
\end{frame}
\begin{frame}{latexmkとは}
  \begin{itemize}
    \item ファイルを保存するだけでタイプセット
    \item 自動でPDFを開いてくれる
    \item 画像の差し替えにも対応
  \end{itemize}
\end{frame}
\begin{frame}{latexmkの導入}
  さっそくlatexmkの導入と行きたいところですが...
    \onslide<2>{\begin{center}
    \Large 実はtexlive2016/2017は導入済み
    \end{center}}
\end{frame}
\begin{frame}{使い方}
  ホームディレクトリに設定ファイル(.latexmkrc)を設置して\\
  コマンドを走らせるだけ
\end{frame}
\end{document}
